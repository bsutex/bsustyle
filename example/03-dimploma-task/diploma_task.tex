\documentclass[a4papper,11pt]{report}

\usepackage{bsumain}
\usepackage{libertine}


\begin{document}
    \begin{center}
        Белорусский государственный университет

        \vspace{2em}

        Кафедра математического моделирования и анализа данных

        \vspace{2em}

        \begin{minipage}[h]{0.7\textwidth}
        Утверждаю

        Заведующий кафедрой \rule{3cm}{0.4pt} Харин Ю.С.
        \end{minipage}

        \vspace{2em}

        \textbf{\Large{ЗАДАНИЕ НА ДИПЛОМНУЮ РАБОТУ}}
    \end{center}

    \vspace{2em}

    Обучающемуся(студенту) \textit{Архангельскому И.А}.

    \begin{enumerate}
    \item Тема дипломной работы

    \textit{Анализ изображений на основе вейвлетов добеши}

    Утверждена приказом ректора БГУ от \rule{2cm}{0.4pt}№\rule{1cm}{0.4pt}

    \item Исходные данные к дипломной работе
        \begin{enumerate}
            \item Добеши И. Десять лекций по вейвлетам. \textit{// Ижевск: НИЦ ``Регулярная и хаотическая динамика''}
            \item Morettin P. A wavelet analysis for time series \textit{// Journal of Nonparametric Statistics}
            \item Абрамович М.С., Мицкевич М.Н. Критерии обнаружения разладок бинарных последовательностей основанные на вейвлет-преобразовании \textit{// Вестник БГУ}
        \end{enumerate}

    \item Перечень подлежащих разработке вопросов или краткое содержание расчетно-пояснительной записки:

        С использованием среды программирования R осуществить:
        \begin{enumerate}
            \item реализацию и сравнительный анализ методов обнаружения разладок бинарных последовательностей предложенных в указанных источниках.
            \item реализацию метода обнаружения разладок бинарных последовательностей основанного на вейвлет преобразовании с использованием вейвлетов Добеши высших порядков.
            \item реализацию и сравнительный анализ методов фильтрации сигналов основанных на вейвлет преобразовании с использованием вейвлетов Добеши высших порядков.
            \item реализацию интерактивной системы иллюстрирующей методы фильтрации сигналов.
            \item сравнительный анализ методов фильтрации в применении к чернобелым изображениям.
        \end{enumerate}
    \item Перечень графического материала (с точным указанием обязательных чертежей и графиков)
    \item Консультанты по дипломной работе с указанием относящихся к ним разделов \textit{Лобач В.И.}
    \item Примерный график выполнения дипломной работы
        \begin{itemize}
            \item 31 марта 2015 г. промежуточный отчет
            \item 28 апреля 2015 г. промежуточный отчет
            \item 12 мая 2015 г. доклад о проделанной работе
        \end{itemize}
    \item Дата выдачи задания \rule{5cm}{0.4pt}

    \end{enumerate}


    Руководитель \rule{6cm}{0.4pt} Лобач В.И.

    Подпись обучающегося \rule{6cm}{0.4pt}

    Дата


\end{document}
